\documentclass[a4paper,11pt]{article}
\usepackage[dutch]{babel}
\usepackage{amsmath}
\usepackage{listings}
\usepackage{graphicx}
\usepackage{a4wide}
\usepackage{xltxtra}

\setlength{\parindent}{0em}
\setlength{\parskip}{1em}

\setmainfont{Bitstream Vera Serif}

\title{Project Algoritmen en Datastructuren III}
\author{Jasper Van der Jeugt}
\date{\today}

\begin{document}

\maketitle
\tableofcontents

\section{Structuur van de code}

\subsection{Verschillende implementaties, dezelfde interface}

De opgave bestaat eruit om verschillende tools te schrijven met hetzelfde nut,
enkel het gebruikte algoritme is verschillend. Om zoveel mogelijk
code-duplicatie te vermijden, scheiden we daarom de implementatie van de
interface.

Een abstracte interface wordt beschreven in \verb#src/search.h#. Elk algoritme
dient deze interface te implementeren. De interface bevat volgende methoden:

\begin{enumerate}
    \item \verb#search_create#: Initialiseert data voor doorzoeken. Afhankelijk
    van het algoritme zal deze functie een aantal variabelen initialiseren.
    \item \verb#search_buffer#: Doorzoek \'e\'en buffer naar het gegeven
    pattern.
    \item \verb#search_free#: Ruim op na het zoeken. Hier kan het algoritme
    bepaalde variabelen dealloceren.
\end{enumerate}

Deze interface laat toe om elk gevraagd zoekalgoritme te implementeren.

\subsection{Buffering}

Bla bla bla

\end{document}
