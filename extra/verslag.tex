\documentclass[a4paper,11pt]{article}
\usepackage[dutch]{babel}
\usepackage{amsmath}
\usepackage{listings}
\usepackage{graphicx}
\usepackage{a4wide}
\usepackage{xltxtra}

\setlength{\parindent}{0em}
\setlength{\parskip}{1em}

\setmainfont{Bitstream Vera Serif}

\title{Project Algoritmen en Datastructuren III}
\author{Jasper Van der Jeugt}
\date{\today}

\begin{document}

\maketitle
\tableofcontents

\section{Structuur van de code}

\subsection{Verschillende implementaties, dezelfde interface}

De opgave bestaat eruit om verschillende tools te schrijven met hetzelfde nut,
enkel het gebruikte algoritme is verschillend. Om zoveel mogelijk
code-duplicatie te vermijden, scheiden we daarom de implementatie van de
interface.

Een abstracte interface wordt beschreven in \verb#src/search.h#. Elk algoritme
dient deze interface te implementeren. De interface bevat volgende methoden:

\begin{itemize}
    \item \verb#search_create#: Initialiseert data voor doorzoeken. Afhankelijk
    van het algoritme zal deze functie een aantal variabelen initialiseren.
    \item \verb#search_buffer#: Doorzoek \'e\'en buffer naar het gegeven
    zoekpatroon.
    \item \verb#search_free#: Ruim op na het zoeken. Hier kan het algoritme
    bepaalde variabelen dealloceren.
\end{itemize}

Deze interface laat toe om elk gevraagd zoekalgoritme te implementeren.

\subsection{Werken met buffers}

Omdat \'e\'en van de vereisten is dat het doorzoeken van grote files mogelijk
is, gebruiken we buffers. In deze subsectie argumenteren we aan dat we een
willekeurig exact string matching algoritme dat werkt op een simpele array
gemakkelijk kunnen uitbreiden tot een algoritme dat op grote files werkt via
buffers.

Onze strategie verloopt als volgt:

\textbf{Gegeven}:

Een initieel lege buffer $b$ met $b_{size}$ de grootte van deze buffer, en
$b_{used}$ het aantal gebruikte tekens in deze buffer. Laat $p$ ons zoekpatroon
zijn met $p_{size}$ de lengte van dit zoekpatroon. Verder is $t$ de te
doorzoeken tekst (met lengte $t_{size}$) en $a$ het gebruikte algoritme.

\textbf{Er geldt}:

\begin{itemize}
    \item initieel is de buffer leeg, dus $b_{used} = 0$;
    \item het patroon moet in de buffer passen, dus $p_{size} \leq b_{size}$.
\end{itemize}

\textbf{Algoritme}:

\begin{itemize}
    \item Zolang we het einde van de te doorzoeken tekst niet hebben bereikt,
    dus zolang $t$ niet leeg is:
    \begin{itemize}
        \item Verplaats de eerste $b_{size} - b_{used}$ tekens van $t$ naar de
        buffer $b$, de eerste $b_{used}$ tekens van $b$ blijven dus behouden.
        \item Doorzoek $b$ met het gegeven algoritme $a$.
        \item Indien er een match begint op \'e\'en van de
        laatste $p_{size} - 1$ posities in de buffer, kan deze niet gedetecteerd
        worden. Daarom verplaatsen we nu de laatste $p_{size} - 1$ tekens naar
        het begin van $b$. Nu geldt dat $b_{used} = p_{size} - 1$.
    \end{itemize}
\end{itemize}

Er is dus telkens een zeker overlap van $p_{size} - 1$ tekens, die (afhankelijk
van het algoritme) eventueel 2 keer doorzocht wordt. Een slechtste geval analyse 
geeft ons dat de maximale overhead gegeven wordt door:

\begin{equation*}
\left( \lceil \frac{t_{size}}{b_{size}} \rceil - 1 \right)
    \cdot \left( p_{size} - 1 \right)
\end{equation*}

Aangezien $p_{size}$ en $t_{size}$ vastliggen, is $b_{size}$ de enige factor
waarmee we de overhead kunnen be\"invloeden. Om de overhead te minimaliseren,
moeten we $b_{size}$ zo groot mogelijk kiezen.

\end{document}
